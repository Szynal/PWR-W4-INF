\documentclass[11pt]{report}
\usepackage{graphicx}
\usepackage{subcaption}
\usepackage[T1]{fontenc}
\usepackage[polish]{babel}
\usepackage{mathptmx}
\usepackage[utf8]{inputenc}
\usepackage{lmodern}
\usepackage{enumitem}
\usepackage{setspace}
\usepackage{advdate}
\usepackage{fancyvrb}
\usepackage[none]{hyphenat}
\usepackage{amsmath}
\usepackage{listings}
\usepackage{color}
\usepackage{floatrow}
\usepackage[left=2.5cm,top=2.5cm,right=2.5cm,bottom=2.5cm,bindingoffset=0.5cm]{geometry}
\usepackage[table,xcdraw]{xcolor}

\usepackage{listings}
\lstset{
    basicstyle=\ttfamily,
    columns=fixed,
    fontadjust=true,
    basewidth=0.5em,
    showstringspaces=false.
    language=VHDL,
    extendedchars=\true,
    literate={ą}{{\k{a}}}1
             {Ą}{{\k{A}}}1
             {ę}{{\k{e}}}1
             {Ę}{{\k{E}}}1
             {ó}{{\'o}}1
             {Ó}{{\'O}}1
             {ś}{{\'s}}1
             {Ś}{{\'S}}1
             {ł}{{\l{}}}1
             {Ł}{{\L{}}}1
             {ż}{{\.z}}1
             {Ż}{{\.Z}}1
             {ź}{{\'z}}1
             {Ź}{{\'Z}}1
             {ć}{{\'c}}1
             {Ć}{{\'C}}1
             {ń}{{\'n}}1
             {Ń}{{\'N}}1
}

\setcounter{chapter}{1}
\begin{document}

\begin{titlepage}
    \newcommand{\HRule}{\rule{\linewidth}{0.5mm}}
    \center
    \includegraphics[scale=0.4]{logo.jpg}\\[1cm] 
    \HRule \\[0.8cm]
    { \Large \bfseries Układy cyfrowe i systemy wbudowane 2 - Projekt}\\[0.4cm]
    \HRule \\[1.5cm]
    { \Large \bfseries Organy wykonane na \textit{Spartanie-3E}}\\[0.4cm]
    \textbf{ }
    \vspace{10mm} 
    
    \begin{minipage}{0.4\textwidth}
    \begin{flushleft} \large
    \end{flushleft}
    \end{minipage}
    \begin{minipage}{0.5\textwidth}
    \begin{flushright} \large
    \vfill
    \vspace{80mm} % mm vertical space
    \par Wydział Elektroniki
    \par Kierunek: Informatyka
    \vfill
    \vspace{10mm} % mm vertical space
    
    \par Skład grupy projektowej:\\
    \par Jakub \textsc{Sokołowski 226080}\\
    \par Paweł \textsc{Szynal 226026} \\
   \vspace{10mm} % mm vertical space
    \par Prowadzący: dr inż. Jarosław Sugier
    \end{flushright}
    \end{minipage}\\[3cm]
   %  {\large Wrocław 2019 r.}\\[2cm]
\end{titlepage}
\tableofcontents
\listoffigures
\lstlistoflistings
\newpage
\section{Wprowadzenie}
\subsection{Temat Pracy}
\centerline{\textbf{ Organy wykonane na \textit{Spartanie-3E}}}


\subsection{Cel projektu}
Założeniem projektu było wykonanie prostych organów z wykorzystaniem układu Spartan3E w celach dydaktycznych.  Jako grupa zdecydowaliśmy się na próbę implementacji jedno-gamowego piania obsługiwanego przez klawiaturę podłączaną do układu, z możliwością zapisywania i odtwarzania danego fragmentu na pamięci RAM urządzenia. Do odtwarzania naszej muzyki wykorzystaliśmy przetwornik DAC i podłączony do niego głośnik.\\\\Prace nad projektem podzieliliśmy na pomniejsze etapy, tj.:
\begin{itemize}
\item Generacja sygnału 1,5 kHz,
\item Wyprowadzenie sygnału z modułu Podzielnika Częstotliwości,
\item Obsłużenie przetwornika DAC,
\item Obsługa klawiatury jako wejścia,
\item Implementacja pamięci RAM,
\item Odtwarzanie z pamięci,
\item Nagrywanie, odgrywanie.
\end{itemize}

\subsection{Opis Sprzętu i Narzędzi}

Wykorzystywany sprzęt: 

\begin{itemize}
    \item Spartan-3E FPGA Starter
	    Jest to w pełni funkcjonalny układ FPGA firmy Digilent. Wyposażony jest w 32-bitowy procesor RISC i interfejsy DDR. Płyta posiada również interfejsy programowania równoległego Xilinx Platform. Płyta jest w pełni kompatybilna ze narzędziami, Xilinx ISE® i ModelSim XE.
    
    \item Klawiatury komputerowej 
        Podłączona do zestawu na porcie PS2
        
    \item Głośnik firmy Tonsil 
        Połączony do pinów GND i DAC-C
\end{itemize}
 
Wykorzystywane narzędzia:
\begin{itemize}
    \item Xilinx ISE®
	    Jest to zintegrowane środowisko programistyczne służące do wykonywania wszystkich operacji związanych z przygotowaniem projektu w układzie FPGA 
    
    \item ModelSim XE
        Symulator, który oblicza i przedstawia graficznie wygenerowany przez układ odpowiedzi na portach wyjściocwych.

\end{itemize}
 
\subsection{Opis interfejsów i używanych protokołów}
	Jak już zostało wspomniane powyżej do interakcji z Naszym układem wymagana jest klawiatura komputerowa z wtyczką PS2 oraz głośnik. Parametry klawiatury czy głośnika, jak i ich właściwości nie mają dla poprawnej pracy układu żadnego znaczenia. W przypadku klawiatury wynika to z standardu jakim jest PS2, a w przypadku głośnika, jest to standard przetwornika DAC. Tym samym nie ma potrzeba zamieszczania danych katalogowych opisywanych elementów.
	
\section{Realizacja Projektu}
\subsection{Zajęcia nr 1}
\par Omówienie pracy i BHP

\subsection{Zajęcia nr 2}

\par Pierwszym celem naszego projektu było zaznajomienie się z modułem DAC i Sawtooth aby za pomocą ich wygenerować falę piło-kształtna o zadanej częstotliwości.

\subsubsection{Sygnały zegarowe}
    \par  Głównym i zarazem podstawowym wykorzystywanym układ jest generator częstotliwości służącej taktowaniu układu. 
    W naszym projekcie korzystaliśmy z podstawowego sygnału zegarowego o częstotliwości 50 MHz. 

    \par W oparciu o podstawowy zegarowy sygnał wejściowy i częstotliwość dźwięku, tworzony był kolejny sygnał zegarowy.
\newpage

\subsubsection{Przetwornik i Moduł DAC}
    \par  W ramach przygotowania do projektu zaznajomiliśmy się z modułem DAC. Jak sama nazwa wskazuje jest to przetwornik cyfrowo-analogowy (z ang. Digital to Analog Converter, DAC) Innymi słowy jest to układ przetwarzający dyskretny sygnał cyfrowy na równoważny mu sygnał analogowy.
   Spartan-3E zawiera kompatybilny czterokanałowy, szeregowy konwerter cyfrowo-analogowy.  
    \vspace{5mm}
    Komunikacja SPI:
    \par  FPGA wykorzystuje interfejs komunikacji szeregowej (SPI -szeregowy interfejs urządzeń peryferyjnych) do czterech kanałów DAC. Magistrala SPI jest w pełni dupleksowa, synchroniczna. Nasz przetwornik ma 12-bit rozdzielczość i korzysta z SPI jako czterokanałowego sposobu przesyłu danych. 

    \vspace{10mm}
    
     Zobrazowanie SPI:

    \begin{figure}[h]
    	\centering
    	\includegraphics[width=0.7\linewidth]{spi.png}
    	\caption{SPI}
	    \label{fig:ps2}
    \end{figure}
    
    \begin{itemize}
	    \item MOSI (ang. Master Output Slave Input) – dane dla układu peryferyjnego,
	    \item MISO (ang. Master Input Slave Output) – dane z układu peryferyjnego, 
	    \item SClk (ang. Serial CLock) – sygnał zegarowy (taktujący), 
	    \item SS (ang. Slave Select) - wybór układu podrzędnego
	\end{itemize}
	\vspace{5mm}
    
     \par W naszym wypadku to SPI działa troszkę inaczej. Dodane są linie DAC\_CS i DAC\_CLR. \newline CS oznacza początek przesyłu(dla stanu wysokiego), a CLR to typowy Clr. SPI przesyła 12-bitowe dane, potem 4 bity określający kanał przetwornika DAC na który ma to trafić.
 
	\vspace{10mm}

    \begin{lstlisting}[label=UCF Location Constraints,caption=UCF Location Constraints,frame=tb,language=vhdl]
        NET "SPI_MISO" LOC = "N10" | IOSTANDARD = LVCMOS33 ;
        NET "SPI_MOSI" LOC = "T4" | IOSTANDARD = LVCMOS33 | SLEW = SLOW | DRIVE = 8 ;
        NET "SPI_SCK" LOC = "U16" | IOSTANDARD = LVCMOS33 | SLEW = SLOW | DRIVE = 8 ;
        NET "DAC_CS" LOC = "N8" | IOSTANDARD = LVCMOS33 | SLEW = SLOW | DRIVE = 8 ;
        NET "DAC_CLR" LOC = "P8" | IOSTANDARD = LVCMOS33 | SLEW = SLOW | DRIVE = 8 ;
    \end{lstlisting}
    \newpage
    
    
\subsection{Zajęcia nr 3}

\subsubsection{Sawtooth Wave}
	
	Fala piło-kształtna jest rodzajem fali niesinusoidalne zaś jest tak nazwana w oparciu o jego podobieństwo do zębów piły. Moduł przesyła odpowiednio przygotowany sygnał piło-kształtny, który następnie poprzez działanie modułu DAC wygrywany jest na głośniku dlatego można powiedzieć że dla generacji dźwięku jest to najważniejszy moduł ze wszystkich. 
	
    Wykorzystywany przez nas wzór do wygenerowania piły:
$$x(t)=t-\underbrace{\lfloor t\rfloor}$$


\begin{figure}[h]
    \paragraph{Schemat}
    	\centering
    	\includegraphics[width=1.1\linewidth]{sawTooth.png}
    	\caption{Schemat testowego sawTooth }
	    \label{fig:ps2}
\end{figure}

	Devider jest modułem testowym przed wykonaniem organków w który odpowiadał za opdzielenie 50MHz na inne częsttliwości. Idea dość prosta i sprowadzająca się do prostego rachunku. 50MHz daje nam 20ns okresu zegarowego.

\begin{lstlisting}[label=Clock_processs,caption= Clock\_processs w module Divider,frame=tb,language=vhdl]
      Clock_processs: process
    	begin
    		Clock <= '0';
    		wait for Clock_period/2;
    		Clock <= '1';
    		wait for Clock_period/2;
    	end process;
    	
    	% Clock_period zostalo zadeklarowane przed tym jako stala 
    	% constant Clock_period : time := 20ns;
    \end{lstlisting}
    
	\begin{figure}[h]
       	\centering
    	\includegraphics[width=1.1\linewidth]{Clock_period.png}
    	\caption{Clock\_period}
	    \label{fig:ps2}
\end{figure}
\newpage

\subsubsection{Dzielenie częstotliwości}

\begin{lstlisting}[label=Dividing,caption= Dzielenie częstotliwości,frame=tb,language=vhdl]
        process( ClkIN, ClrIN, temp )
        begin
        	if(ClrIN = '1') then 
        		iterator <= 1;
        		temp <= '0';
        	elsif rising_edge(ClkIN)  then
        		iterator <= iterator + 1;
        		if (iterator = frequency) then
        			temp <= NOT temp;
        			iterator <= 1;
        		end if;
        	end if;
        ClkOUT <= temp;
        end process;
    \end{lstlisting}
    
	\begin{figure}[h]
       	\centering
    	\includegraphics[width=1.1\linewidth]{test2.png}
    	\caption{Test dzielenia częstotliwości}
	    \label{fig:ps2}
\end{figure}

\subsubsection{Generacja piły}

\begin{lstlisting}[label=saw,caption= Generacja piły,frame=tb,language=vhdl]
        process(ClockIN, ClrIN)
        begin
            if(ClrIN = '1') then
                iteratorProbek <= 0;
            elsif(rising_edge(ClockIN)) then
                if(iteratorProbek = 63) then
                    iteratorProbek <= 0;
                else
                    iteratorProbek <= iteratorProbek + 1;
                end if;
            end if;
        end process;
    \end{lstlisting}
    
	\begin{figure}[h]
       	\centering
    	\includegraphics[width=1.1\linewidth]{pila.png}
    	\caption{Generacja piły}
	    \label{fig:ps2}
\end{figure}
\newpage
\subsection{Zajęcia nr 4}

	Kolejnym zadaniem naszego projektu było zaimplementowanie modułów potrzebnych do obsługi organków. Potrzebowaliśmy do tego zadania następujących modułów:
	
		\begin{itemize}
    	    \item Moduł będącego odbiornikiem kodów wysyłanych przez klawiaturę PS/2,
        	\item Moduł obsługujący jedną gamę na podstawie klawiszy,
    	    \item Moduł generujący dźwięki,
    	    \item Moduł przeznaczony jest do wyprowadzeń zewnętrznych FPGA-naszego głośnika.
       	\end{itemize}
	
\subsubsection{Moduł PS2\_Kbd}

\par Moduł ten jest odbiornikiem kodów wysyłanych przez klawiaturę PS/2. Został on pobrany ze strony http://www.zsk.ict.pwr.wroc.pl

    \begin{figure}[h]
    	\centering
    	\includegraphics[width=0.7\linewidth]{ps2.png}
    	\caption{Moduł Ps2}
	    \label{fig:ps2}
    \end{figure}

    \begin{itemize}
		\item DO(7:0) - Przechowuje wartość odbieranego kodu.
	    \item E0 - Flaga sygnalizująca czy wchodzący sygnał był poprzedzony bajtami X”E0” (tzw. kod rozszerzony).
	    \item F0 - Flaga sygnalizująca czy nastąpiło zwolnienie naciśniętego klawisza.
	    \item Sygnał DO\_Rdy - Sygnalizuje zakończenie odbioru kodu (impuls jednotaktowy).
	\end{itemize}
	
\newpage
\subsubsection{Moduł PianoKey}
     Zadaniem tego modułu jest obsługa maszyny stanów klawiszy pianina. Po podaniu zbocza rosnącego na wejście \textit{K\_rdy} zaczytywany jest kod skaningowy klawiatury. W zależności od wciśniętego klawisza, ustawiany jest odpowiedni stan. Oprogramowana została jedna oktawa tzn. 12 klawiszy co daje nam 12 stanów odpowiadających wciśniętym klawiszom, oraz jeden stan odpowiadający braku wciśniętego klawisza - stan spoczynku. Klawisze klawiatury które odpowiadają dźwiękom pianina to (A, W, S, E, D, F, T, G, Y, H ,U , J). Nazwy stanów odpowiadają wciśniętym klawiszom. Po wciśnięciu klawisza, którego należy do obsługiwanej oktawy następuje przejście do stanu (\textit{next\_state} w kodzie) odpowiadającego klawiszowi. Jeśli wciśnięty zostanie nieobsługiwany klawisz, maszyna pozostaje w stanie spoczynku (oznaczonym jako \textit{DEF}). Przejście w stan w spoczynku odbywa się również po puszczeniu klawisza. Puszczenie klawisza sygnalizuje wejście \textit{F0}. Zarządzanie stanami wykonuje proces \textit{StMch}, którego kod znajduje się na listingu \ref{piano-stmch}.
   
	Ponieważ po stanie wciśnięciu klawisza zawsze występuje stan spoczynku, na takim pianinie nie jest możliwe granie akordów - naciśnięcie kilku klawiszy jednocześnie.
   
    \begin{figure}[h]
    	\centering
    	\includegraphics[width=0.4\linewidth]{Pianokey.png}
    	\caption{Moduł PianoKey}
	    \label{fig:piano}
    \end{figure}

	Wejścia:
	\begin{itemize}
    	\item Clr - reset,
    	\item Clk – systemowy zegar,
    	\item K\_rdy – odebranie sygnału o odczytaniu klawisza z klawiatury na porcie PS2,
    	\item IN\_klaw – kod skaningowy klawisza odczytanego z klawiatury,
    	\item F0 – kod zwolnienia klawisza.
	\end{itemize}
	Wyjścia:
	\begin{itemize}
		\item FreqOUT – wyjście, będące tłumaczeniem wciśniętego klawisza na konwencję używaną dalej.
	\end{itemize}
	
\begin{lstlisting}[label=piano-stmch,caption=Fragment procesu StMch,frame=tb,language=vhdl]
StMch : process( state, In_key, K_rdy, tmp )
   begin
      next_state <= state; 
      case state is      
         when Def =>
            if K_rdy = '1' and In_key = X"1C" then
               next_state <= A;
            .....
            
            elsif K_rdy = '1' then
               next_state <= DEF;
            end if;
         when A =>
            if F0 = '1' then
               next_state <= DEF;
            end if;
            .....
         when K =>
            if F0 = '1' then
               next_state <= DEF;
            end if;
      end case;
end process StMch;
\end{lstlisting}
W zależności od obecnego stanu, należy wygenerować sygnał wyjściowy \textit{FreqOut} odpowiadający wciśniętemu klawiszowi. Wartość 0 oznacza stan spoczynku - ciszę, a wartości 1-12 kolejne klawisze - dźwięki.  Do zakodowania 13 różnych stanów potrzeba co najmniej 4 bitów, więc sygnał wyjściowy \textit{FreqOut} jest 4-bitowym wektorem. Proces konwertujący stan na sygnał jest przedstawiony na listingu \ref{piano-outpr}.
\begin{lstlisting}[label=piano-outpr,caption=Fragment procesu OutPr,frame=tb,language=vhdl]
OutPr : process ( state )
   begin
      case state is
         when A =>
            FreqOUT <= conv_std_logic_vector(1,4);
         when W =>
            FreqOUT <= conv_std_logic_vector(2,4);
        
        ...
        
         when DEF =>
            FreqOUT <= conv_std_logic_vector(0,4);
         end case;
   end process OutPr;
\end{lstlisting}

\newpage

	\subsubsection{SawToothGenerator} 
	
	Jest to najważniejszy moduł ze wszystkich. Odpowiedzialny jest jest za  generowanie dźwięków za pomocą wygenerowanego wektora sygnału piło-kształtnego.
	
	
    \begin{figure}[h]
    	\centering
    	\includegraphics[width=0.5\linewidth]{sawToothGenerator.png}
    	\caption{Moduł SawToothGenerator}
	    \label{fig:sawtoothgen}
    \end{figure}
	

    \vspace{10mm} 
	Wejścia:
	\begin{itemize}
	\item Clk – wejście zegara systemowego,
	\item Clr - reset,
	\item FreqIN – wejście kodu oznaczające częstotliwość dźwięku.
	\end{itemize}
	
    \vspace{10mm} 
	Wyjścia:
	\begin{itemize}
	\item AddrOUT – wyjście addresowe,
	\item CmdOUT – wyjście rozkazu,
	\item StartOUT – wyjście Start,
	\item SawtoothOUT – wektor sygnału piłokształtnego.
	\end{itemize}
	
	Powyższy moduł składa się z trzech procesów:
    \begin{itemize}
	\item SetFreq, 
	\item FreqDiv,
	\item SawToothGen,
	\end{itemize}
	\newpage
	
	\paragraph{SetFreq} 
	Pierwszy zaimplementowany proces-SetFreq jest szeregiem operacji warunkowych-elsif odpowiedzialnych za przypisanie częstotliwości w zależności od wciśniętego przycisku na klawiaturze. Wejściowy FreqIN przechowuje wartość klawisza zaś Freq ma przypisaną wartość oznaczającą częstotliwość konkretnego dźwięku, zgodną z oktawą na Pianie. \cite{4}
    Na listingu \ref{setfreq} znajduje się proces który przekształca kod dźwięku na jego częstotliwość. Częstotliwość dźwięku generowana poprzez sumowanie impulsu piły o częstotliwości 1,5 Hz.  Przykładowo, jeśli chcemy uzyskać za pomocą sygnału 1.5 hz dźwięk \textit{C2} który ma częstotliwość 131 Hz należy dodać 1.5 Hz 87 razy. Analogicznie, w najwyższym dźwięku oktawy - \textit{C3} o częstotliwości 262 Hz znajdą się 173 składowe 1.5 Hz ($262/1.5\approx173$). Liczba ile razy należy dodać 1.5 Hz by otrzymać dany dźwięk jest w kodzie oznaczana jako \textit{Freq}. W tym projekcie oprogramowano klawisze od \textit{C2} do \textit{C3} o częstotliwościach od 131 do 262 Hz - odpowiadający im zakres \textit{Freq} to 87-173.
    
    
    \vspace{10mm} 
    \begin{lstlisting}[label=setfreq,caption=Proces SetFreq,frame=tb,language=vhdl]
    SetFreq: process(FreqIN, Freq)
    begin
        if(FreqIN = "0001") then 
        	Freq <= 173; 
        elsif(FreqIN = "0010") then
        	Freq <= 164; 
        elsif(FreqIN = "0011") then
        	Freq <= 154;
        elsif(FreqIN = "0100") then
        	Freq <= 146;
        elsif(FreqIN = "0101") then
        	Freq <= 137;
        elsif(FreqIN = "0110") then
        	Freq <= 130;
        elsif(FreqIN = "0111") then
        	Freq <= 123;
        elsif(FreqIN = "1000") then
        	Freq <= 116;
        elsif(FreqIN = "1001") then
        	Freq <= 109;
        elsif(FreqIN = "1010") then
        	Freq <= 103;
        elsif(FreqIN = "1011") then
        	Freq <= 97;
        elsif(FreqIN = "1100") then
        	Freq <= 92;
        elsif(FreqIN = "1101") then
        	Freq <= 87;
        else
        	Freq <= 0;
        end if;
    end process;
    \end{lstlisting}
    \newpage

	\paragraph{FreqDiv} 
	
    \par Zadaniem procesu FreqDiv jest w oparciu o ustaloną częstotliwość, i zegarowy sygnał wejściowy utworzenie sygnału zegarowego o częstotliwości zgodnej z parametrem Freq będący wyjśćiem procesu SetFreq. Dodatkowo w tym samym procesie, tj. FreqDiv opisywany jest sygnał StartOUT, którego obecność wynika z modułu DACWrite. StartOUT jest  sygnałem jednotaktowym dla zegara systemowego wygenerowanym za każdym razem jak zegar o częstotliwości wejściowej FreqIN zmieni swój stan.


    \vspace{20mm}         	
    \begin{lstlisting}[label=freqdiv,caption=Proces FreqDiv,frame=tb,language=vhdl]
    FreqDiv: process( Clk, Clr, tmpFreqDiv )
    begin
        if(Clr = '1' ) or (Freq = 0) then 
            iFreqDiv <= 1;
            tmpFreqDiv <= '0';
            StartOUT <= '0';
    	elsif rising_edge(Clk)  then
            if (iFreqDiv = Freq) then
                tmpFreqDiv <= NOT tmpFreqDiv;
                iFreqDiv <= 1;
                if (stCnt mod 2 = 0) then
                    StartOUT <= '1';
                    stCnt <= stCnt + 1;
                else
                    stCnt <= stCnt + 1;
                end if;
            else
                iFreqDiv <= iFreqDiv + 1;
                StartOUT <= '0';
            end if;
        end if;
    ClkSawTooth <= tmpFreqDiv;
    end process;
    \end{lstlisting}
    
    	
    \begin{figure}[h]
    	\centering
    	\includegraphics[width=1.1\linewidth]{test3.png}
    	\caption{Test dzielenia częstotliwości}
	    \label{fig:FreqDiv}
    \end{figure}
    
    \newpage
    
	\paragraph{SawToothGen} 	  
	
	Jest to ostatni proces w naszym module SawToothGenerator. Zadaniem tego procesu jest utworzonie sygnału piłokształtnego zgodnego z sygnałem zegarowym z poprzedniego procesu-FreqDiv. Sygnał piłokształtny generowany jest na zasadzie inkrementowanego wektora, który po 64 okresie się resetuje z powrotem. 
	
    \vspace{10mm} 
    	
    \begin{lstlisting}[label=sawtoothgen,caption=Process: FreqDiv,frame=tb,language=vhdl]
    SawToothGen: process( ClkSawTooth, Clr )
    begin
        if(Clr = '1') then
            iSawGen <= 0;
        elsif(rising_edge(ClkSawTooth)) then
            if(iSawGen = 63) then
                iSawGen <= 0;
            else
                iSawGen <= iSawGen + 1;
            end if;
        end if;
    end process;
    \end{lstlisting}
    	
	
    \vspace{10mm} 
	
	\paragraph{Sygnały wyściowe} 
	
	W ostatniej części tego modułu jest wysyłanie sygnałów wyjściowych CmdOUT i AddrOUT. Sygnały te są potrzebne dla modułu DACWrite. Tutaj także wysyłany jest sygnał SawToothOUT będący sygnały piłokształtnym z poprzedniego procesu po operacji konkatenacji z sześcioma zerami umieszczonym na najmniej ważnych pozycjach. Operacja konkatenacji wynika ponownie z potrzeby dostarczenia tego typu sygnału do modułu DACWrite.
	
    \vspace{10mm} 
        
    \begin{lstlisting}[label=sawtooth-in,caption=Sygnały wyściowe,frame=tb,language=vhdl]
    
    CmdOUT <= conv_std_logic_vector(Cmd,4);
    AddrOUT <= conv_std_logic_vector(Addr,4);
    SawtoothOUT <= conv_std_logic_vector(iSawGen,6) & "000000";
    \end{lstlisting}
    \newpage

\subsubsection{Moduł DAC}
  \par Niezbędny do realizacji naszego projektu było zaznajomienie się z przetwornikiem cyfrowo-analogowym (DAC), który konwertuje sygnał cyfrowy na sygnał analogowy. Był on kluczowy dla naszych organów ponieważ przetwornik DAC jest powszechnie używany w odtwarzaczach muzycznych do konwersji cyfrowych strumieni danych na analogowe sygnały audio.
    Obsługa przetwornika DAC odbywa się za pomocą magistrali SPI.

	
    \par Moduł DacWrite został pobrany ze strony Zespołu Systemów Komputerowych i był odpowiedzialny za komunikacje z przetwornikiem cyfra/analog LTC264 w naszym układzie. W zależności od stanu wejścia \textit{Start}, dane z wejść Cmd, Addr i Data są zatrzaskiwane i przesyłane szeregowo.
    
    \vspace{10mm} 
    \begin{itemize}
		\item{Start - zatrzaskuje wartości na pozostałych wejściach}
	    \item{Cmd(3:0) - polecenie jakie można wydać przetwornikowi, w tym wypadku będzie to polecenie '0011' aktualizuje wybrane wyjście DAC zawartością DATA}
	    \item{Addr(3:0) - adres pinu na który wysyłany jest sygnał. Na to wejście również podawana jest stała wartość '0010' która oznacza pin DAC-C, do którego podłączony jest głośnik}
	    \item{DATA(11:0) - dane przekazywane na pin, w tym wypadku wysokość generowanego dźwięku}
	\end{itemize}

    \begin{figure}[h]
    	\centering
    	\includegraphics[width=0.7\linewidth]{DACWrite.png}
    	\caption{Moduł DACWrite}
	    \label{fig:1}
    \end{figure}
\newpage


\section{Schemat}

\begin{center}
 \includegraphics[angle=270,width=0.45\linewidth]{UCISW2_Schemat.png}
\end{center}
\newpage
    
 \section{Podsumowanie}

\begin{figure}[hbt!]
    	\centering
    	\includegraphics[width=1\linewidth]{status.png}
    	\caption{Stan projektu}
	    \label{fig:1}
    \end{figure}

     \vspace{10mm} % mm vertical space
    
    
    \begin{table}[hbt!]
    \paragraph{Device Utilization Summary }
     \vspace{5mm} % mm vertical space
    \begin{tabular}{|l|l|l|l|}
    \hline
    \rowcolor[HTML]{FFFE65} 
    \multicolumn{1}{|c|}{\cellcolor[HTML]{FFFE65}\textbf{Logic Utilization}} & \multicolumn{1}{c|}{\cellcolor[HTML]{FFFE65}\textbf{Used}} & \multicolumn{1}{c|}{\cellcolor[HTML]{FFFE65}\textbf{Available}} & \multicolumn{1}{c|}{\cellcolor[HTML]{FFFE65}\textbf{Utilization}} \\ \hline
    Number of Slice Flip Flops                                 & 140                                                        & 9,312                                                           & 1\%                                                               \\ \hline
    Number of 4 input LUTs                                     & 187                                                        & 9,312                                                           & 2\%                                                               \\ \hline
    Number of occupied Slices                                  & 142                                                        & 4,656                                                           & 3\%                                                               \\ \hline
    Number of Slices containing only related logic             & 142                                                        & 142                                                             & 100\%                                                             \\ \hline
    Number of Slices containing unrelated logic                & 0                                                          & 142                                                             & 0\%                                                               \\ \hline
    Total Number of 4 input LUTs                               & 226                                                        & 9,312                                                           & 2\%                                                               \\ \hline
    Number used as logic                                       & 187                                                        &                                                                 &                                                                   \\ \hline
    Number used as a route-thru                                & 39                                                         &                                                                 &                                                                   \\ \hline
    Number of bonded IOBs                                      & 14                                                         & 232                                                             & 6\%                                                               \\ \hline
    Number of BUFGMUXs                                         & 2                                                          & 24                                                              & 8\%                                                               \\ \hline
    Average Fanout of Non-Clock Nets                           & 2.74                                                       &                                                                 &                                                                   \\ \hline
    \end{tabular}
    \caption{}
    \label{Device Utilization Summary}
    \end{table}

\newpage
    
\section{Wnioski}

Udało nam się wykonać poprawnie główną część projektu czyli w pełni funkcjonalne jedno-gamowe organki. Część ta wykonana została stosunkowo szybko - organki zagrały pierwszy raz już na czwartych zajęciach. Niestety, po zabraniu się za tworzenie pozytywki, natrafiliśmy na wiele problemów z poprawnym przekierowaniem sygnałów odczytanych z ROM. Funkcja ta wymagała obejścia obecnych modułów i nie mogliśmy sobie z tym poradzić, bez znaczącej degradacji istniejącego kodu. Duży problem też stwarzało to, że praca nad projektem na płytce była możliwa tylko w czasie zajęć - pomysły i rozwiązania na które wpadaliśmy po zakończeniu zajęć mogły być zweryfikowane dopiero dwa tygodnie później.


\begin{thebibliography}{9}		
	\bibitem{1} 
	Spartan-3E FPGA Starter Kit Board User Guide,
	\\\texttt{https://www.xilinx.com/support/documentation/boards\_and\_kits/ug230.pdf}
	\bibitem{2} 
	LTC2624 Manual,
	\\\texttt{http://cds.linear.com/docs/en/datasheet/2604fd.pdf}
	\bibitem{3} 
	Moduł PS2\_Kbd, DACWrite,
	\\\texttt{http://www.zsk.ict.pwr.wroc.pl/zsk\_ftp/fpga}
	\bibitem{4} Tablica częstotliwości dźwięków pierwszej oktawy gamy C-dur,
	\\\texttt{https://www.liutaiomottola.com/formulae/freqtab.htm}
\end{thebibliography}
	
\end{document}